\appendix{TileLink Memory Opcodes}
\label{a.memopcodes}

\begin{table}
\centering
\begin{tabular}{|l|l|l|}
\hline
Name  & Code & Operation \\ \hline \hline
M\_XRD      & b00000 & integer load \\ \hline 
M\_XWR      & b00001 & integer store \\ \hline 
M\_PFR      & b00010 & prefetch with intent to read \\ \hline 
M\_PFW      & b00011 & prefetch with intent to write \\ \hline 
M\_XA\_SWAP & b00100 & atomic swap \\ \hline 
M\_NOP      & b00101 & no op \\ \hline 
M\_XLR      & b00110 & load release \\ \hline 
M\_XSC      & b00111 & store conditional \\ \hline 
M\_XA\_ADD  & b01000 & atomic add \\ \hline 
M\_XA\_XOR  & b01001 & atomic xor \\ \hline 
M\_XA\_OR   & b01010 & atomic or \\ \hline 
M\_XA\_AND  & b01011 & atomic and \\ \hline 
M\_XA\_MIN  & b01100 & atomic min \\ \hline 
M\_XA\_MAX  & b01101 & atomic max \\ \hline 
M\_XA\_MINU & b01110 & atomic min unsigned \\ \hline 
M\_XA\_MAXU & b01111 & atomic max unsigned \\ \hline 
M\_FLUSH    & b10000 & write back dirty data, cede R+W permissions \\ \hline 
M\_PRODUCE  & b10001 & write back dirty data, cede W permissions \\ \hline 
M\_CLEAN    & b10011 & write back dirty data, retain R+W permissions \\ \hline
\end{tabular}
\caption{Operations expressible via the \code{op\_code} field of Acquire transactions.}
\label{tab:opcodes}
\end{table}

\begin{table}
\centering
\begin{tabular}{|l|l|l|}
\hline
Name  & Code & Meaning \\ \hline \hline
MT\_B & b000 & byte \\ \hline
MT\_H & b001 & half \\ \hline
MT\_B & b010 & word \\ \hline
MT\_D & b011 & double \\ \hline
MT\_BU & b100 & byte unsigned \\ \hline
MT\_HU & b101 & half unsigned \\ \hline
MT\_WU & b110 & word unsigned \\ \hline
\end{tabular}
\caption{Operation sizes expressible via the \code{op\_size} field of Acquire transactions.}
\label{tab:opsizes}
\end{table}

Table~\ref{tab:opcodes} lays out the codes for operations
that can be inserted into the \code{op\_code} field of Acquire transactions.
They are derived from the interface between the Rocket pipeline and its data cache.
They correspond to some degree with the RISC-V RV64A memory operations.
Similarly, Table~\ref{tab:opsizes} lays out the codes for expressing the size of operation
that should occur in memory (for atomic operations).

\begin{table}[h]
\begin{center}
\setlength{\tabcolsep}{4pt}
\begin{tabular}{p{1.2in}@{}p{0.8in}@{}p{0.6in}@{}p{0.4in}@{}p{0.2in}@{}p{0.2in}@{}p{0.4in}l}
\\
\instbitrange{168}{41} &
\instbitrange{40}{8} &
\instbitrange{7}{6} &
\instbitrange{5}{4} &
\instbit{3} &
\instbit{2} &
\instbitrange{1}{0} \\
\cline{1-7}
\multicolumn{1}{|c|}{---} &
\multicolumn{1}{c|}{addr} &
\multicolumn{3}{c|}{---} &
\multicolumn{1}{c|}{w} &
\multicolumn{1}{c|}{00} &
Prefetches \\
\cline{1-7}
\\
\cline{1-7}
\multicolumn{1}{|c|}{data} &
\multicolumn{1}{c|}{addr} &
\multicolumn{2}{c|}{size} &
\multicolumn{1}{c|}{a} &
\multicolumn{1}{c|}{w} &
\multicolumn{1}{c|}{01} &
Uncached Get and Puts \\
\cline{1-7}
\\
\cline{1-7}
\multicolumn{1}{|c|}{data} &
\multicolumn{1}{c|}{addr} &
\multicolumn{1}{c|}{size} &
\multicolumn{3}{c|}{aluop} &
\multicolumn{1}{c|}{10} &
Uncached Atomics \\
\cline{1-7}
\\
\cline{1-7}
\multicolumn{1}{|c|}{---} &
\multicolumn{1}{c|}{addr} &
\multicolumn{2}{c|}{size} &
\multicolumn{1}{c|}{u} &
\multicolumn{1}{c|}{w} &
\multicolumn{1}{c|}{11} &
Cached \\
\cline{1-7}
\end{tabular}
\end{center}
\caption{Proposed memory opcode encodings for Acquire and Probe messages.}
\label{tab:memopformats}
\end{table}

