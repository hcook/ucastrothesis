\appendix{Survey of Software-Memory Management Extensions}
\label{a.swmem}
  
SW memory management techniques from commercial architectures can be grouped into three high-level categories. The first category is mechanisms dealing with setting the mode and behavior of the underlying hardware data buffers. The second category is mechanisms for specifying the behavior of particular load and store instructions, including prefetches. The third category is mechanisms for modifying the status of data currently in a particular cache. 

Data buffer management operations are expressed in terms of the actual physical hardware. They include such concrete directives as: deactivate an entire cache, partition some particular ways of some particular cache, change the indexing mode of a cache slice. Once reconfigured by these mechanisms, the data buffer in question behaves differently until a new management directive is supplied.

Load, store and prefetch operations (i.e. data accesses) may take on a variety of additional specifiers that govern their behavior as they interact with the memory hierarchy. Examples of these specifiers include whether capacity for data from this access should be allocated in a particular hierarchy level, what coherence state that data should be in, what replacement policy status that data should have, and whether the access has any particular memory ordering requirements. Prefetches may have additional specifiers not provided for normal loads and stores. These specifiers are optional in many architectures, though VLIW placement flags and vector memory ordering flags are two notable exceptions.

Rather than flagging every instruction with the specifiers explicitly within the opcode, some processors provide modes that govern what specifiers are appended to every load or store issued by the processor while a given mode bit is set. This mode bit setting is similar to how floating point rounding modes are generally implemented. Another option is to append certain specifiers only to instructions whose target address (virtual or physical) falls within a certain range or ranges.

The specifiers for data accesses may be expressed using concrete terms (e.g. ``Put in L1,'' ``Make CC state O,'' ``Make data LRU position 0'') or more abstract terms (e.g. ``Data is streaming,'' ``Data will be read by others soon''). Abstract terms may make code more portable, but are also best suited for cases when functional correctness does not depend on the implementation of the specifier.

The third category is mechanisms for modifying the status of data currently in a particular cache. The directives again address placement (e.g. evict this line), coherence (e.g. clean this line), and replacement policy (e.g. mark this line as next-to-evict). Again, the behavior may be specified concretely (e.g. ``evict from L1'') or abstractly (e.g. ``evict from all private levels''). The target for these modifications can be expressed in terms of individual addresses, address ranges, particular physical subsets of the cache (ways or sets), or just the entire cache.  The concrete mechanisms differ from data buffer management operations in that they target a dataset currently in the cache, rather than governing future data buffer behavior.

In the case of concretely-specified mechanisms, the interface for specifying behavior needs to transparently expose the implementation of memory hierarchy. Even abstract specifications make some assumptions. Since ISAs are designed to be portable across many implementations, most architectures hedge even on concrete mechanisms by stating that the actual behavior is implementation dependent or optional. However, in systems where functionally correct execution is dependent on the behavior of software memory management instructions, such ill-specified behaviors will not be acceptable. Mandatory enforcement of abstracted specifications of behavior are a promising middle-ground between the burdens of portability and the requirements for correct functionality. 

Mandatory vs. optional is another axis on which software memory management instructions can be classified. The two degrees of enforcement correspond with two distinct goals: those instructions that optimize performance, and those instructions that are necessary for correct functionality.
This functionality might be data placement, in which case we can contrast optional prefetches with mandatory VLIW explicit placement flags. Or this functionality might be cache coherence, in which case we might contrast instructions that proactively clean dirty data that is known to be needed by remote processors with abstract load flags that prevent data from being refilled into levels of the hierarchy that are not kept coherent by hardware. While it is acceptable for the former types to be treated as optional hints, the later types must be obeyed.

\section{Background}

\begin{landscape}
\begin{table}[p] 
\begin{small}
\begin{tabular}{|l|c|c|c||c|c|c|c||c|c||c|c|c||c|c||c|c|} 
\hline Feature & 
               {\begin{sideways}Cray-2\end{sideways}} & 
               {\begin{sideways}Cray X-1\end{sideways}} & 
               {\begin{sideways}Cray T3D/T3E\end{sideways}} & 
               
               {\begin{sideways}NVIDIA PTX\end{sideways}} & 
               {\begin{sideways}Larrabee\end{sideways}} & 
               {\begin{sideways}Altivec\end{sideways}} & 
               {\begin{sideways}Intel SSE\end{sideways}} & 
               
               {\begin{sideways}HPL-PD\end{sideways}} & 
               {\begin{sideways}Intel Itaniums\end{sideways}} & 
               
               {\begin{sideways}PA-RISC\end{sideways}} & 
               {\begin{sideways}SPARC\end{sideways}} & 
               {\begin{sideways}POWER\end{sideways}} & 
               
               {\begin{sideways}TI TMS320C6*\end{sideways}} &
               {\begin{sideways}Intel XScale\end{sideways}} & 
               
               {\begin{sideways}Rigel\end{sideways}} & 
               {\begin{sideways}VLS\end{sideways}} 
               \\ \hline \hline
               
                         
 Deactivate entire cache (mode) & & & & & & & & & & & & &\ding{52}& & &\\ \hline
 Allocate capacity to scratchpad (mode) & & & & \ding{52} & \ding{52} & & & & & & & &\ding{52}&\ding{52}&\ding{52}&\\ \hline
 Allocate line of capacity to scratchpad (inst) & & & & & & & & & & & & & &\ding{52}& &\\ \hline \hline

 Separate insts for sep mems &  \ding{52}  & & & \ding{52}& & & & & & & & & & & &\\ \hline
 
 Don't allocate capacity in cache (flag, specific) & & & & & & & &\ding{52}&\ding{52}& & & & & & &\\ \hline
 Don't allocate capacity in cache (flag, L/G) & & \ding{52}  & \ding{52} & \ding{52} & & & \ding{52}& & &  \ding{52}& & & & &\ding{52}&\\ \hline
  Don't allocate capacity in cache (range) & & &\ding{52}& & & & & & & & &  \ding{52}&\ding{52}& & &\\ \hline
   Don't allocate capacity in cache (mode) & & & & & & & & & & & & & &  \ding{52} & &\\ \hline
%Allocate all following blocks with specific CC state (mode) & & & & & & & & & & & & & & & &\\ \hline
 Allocate block with specific CC state (flag) & & \ding{52}  & & & & & \ding{52}& &\ding{52}& & & \ding{52}& & & &\\ \hline
  
 
 Allocate block with specific LRU state (flag) & & & &\ding{52}& & & & & & & & & & & &\\ \hline
 
 Memory ordering requirement (flag) & & \ding{52}  & & & & & & & & & & & & & &\\ \hline \hline

  Prefetch into data cache & & \ding{52} & & \ding{52} & \ding{52} & \ding{52} & \ding{52} & \ding{52}&\ding{52}& & \ding{52}&  \ding{52}& & &\ding{52}&\\ \hline
  Data prefetch into special buffer & & & \ding{52} & & & &\ding{52}&\ding{52}& & & & & & & &\\ \hline
  Pre-allocate capacity for stores & & & \ding{52} & & & & & & & & & & &\ding{52}& &\\ \hline \hline
  
   Lock/unlock data present in entire cache (mode) & & & & & & & & & & & & & &\ding{52}& &\\ \hline
 Lock data placed in cache in future (mode) & & & & & & & & & & & & & &\ding{52}& &\\ \hline
 Lock/unlock data present in cache line (inst) & & & & & & & & & & & & \ding{52} & &\ding{52}& &\\ 
\hline 
  Adjust LRU status of line (inst) & & & & &\ding{52}&\ding{52}& & & & & & & & & &\\ \hline \hline
  
   Invalidate cache line & & & & & \ding{52} & & & &\ding{52}& & & \ding{52}& &\ding{52}&\ding{52}&\\ \hline
   Invalidate cache range & & & & & & & & & & & & &\ding{52}&\ding{52}& & \\ \hline
   Invalidate cache ways & & & & & & & & & & & & & &\ding{52}& & \\ \hline
   Invalidate entire cache & & & & & & & & & & & & &\ding{52}&\ding{52}&\ding{52}& \\ \hline
   Clean cache line & & & & & & & & & & & & \ding{52}& &\ding{52}& &\\ \hline
   Clean cache range & & & & & & & & & & & & &\ding{52}& & & \\ \hline
   Clean entire cache & & & & & & & & & & & & &\ding{52}&\ding{52}& & \\ \hline
   Clean and evict cache line & & & & & & & & & & & & \ding{52}& & & &\\ \hline
   Clean and evict cache range & & & & & & & & & & & & &\ding{52}& & & \\ \hline
   Clean and evict entire cache & & & & & & & & & & & & &\ding{52}& & & \\ \hline
 \end{tabular} 
\end{small}
 \caption{Overview of features in past architectures}
 \end{table}
 \end{landscape}

\subsection{Vector}
\subsubsection{Cray-2}
%http://www.craywiki.com/oldarchive/Manuals/Cray%202%20Computing%20System-%20December%2020,%201982.pdf

Background processors can serve as engines for memory-memory transfers, each with a small local memory for holding operands during the transfer. Local memories used as register files during computation. Separate instructions for accessing local vs. common memory.

\subsubsection{Cray X-1}
%http://docs.cray.com/books/S-2314-51/html-S-2314-51/S-2314-51-toc.html

Vector memory references can have cache hints, expressed as flags on vector ld/st ops:
\begin{itemize}
\item Data should not allocate space in cache if not present
\item New allocations should be in shared state
\item New allocations should be in exclusive state (default)
\end{itemize}

The vrip instruction is used at the end of a sequence of vector instructions to reduce the size of the processor state that must be saved and restored when switching processor contexts, and also to release physical registers in implementations that rename the vector registers.

Scalar memory references include prefetch from from [reg] plus scaled offset or scaled index

Memory ordering flags (G,M,L) can be attached to instructions, interact with barriers.
%http://docs.cray.com/books/S-2314-51/html-S-2314-51/x3724.html

\subsubsection{Cray T3D and T3E}
%ftp://ftp.cray.com/product-info/mpp/T3D_Architecture_Over/T3D.overview.html
%http://docs.cray.com/books/2178_3.0.1/html-2178_3.0.1/z825105374dep.html
% TODO: http://ieeexplore.ieee.org/xpls/abs_all.jsp?isNumber=21417&arNumber=993205&isnumber=21417&arnumber=993205&tag=1
\begin{itemize}
\item Reads: cacheable read, cacheable atomic swap read, cacheable ``readahread'', non-cacheable, non-cacheable atomic swap. ``Readahead'' buffers data in support circuitry to avoid local DRAM access.
\item Writes: cacheable write, uncacheable writes
\item Data prefetch: Transfers from remote memory to prefetch queue in support circuitry, not to cache.
\end{itemize}

\subsection{SIMD/SIMT}

\subsubsection{NVIDIA PTX}

PTX ISA version 2.0 introduced optional cache operators on load and store instructions. Note that in \verb=sm_20= implementations L2 is shared by all SMs, but no hardware coherence is provided amongst L1s.

Cache operators/flags for loads:
\begin{itemize}
\item ``ca'':  Default. Cache at all levels, has temporal locality. Allocates in L1/L2 with normal eviction policy.
\item ``cg'': Cache only at global level (L2). Data bypasses L1. Existing matching lines in bypassed L1 will be evicted.
\item ``cs'': Cache streaming, implying the data lacks temporal locality. Allocates global memory addresses with evict-first policy in L1 and L2. For a local memory addresses, this performs a \verb=ld.lu=.
\item ``lu'': Last use, implying that the line will not be used again. For local addresses, this prevents unecessary writebacks of spilled registers and stack frames by discarding the line from the L1. For global addresses, this performs a \verb=ld.cs=.
\end{itemize}

Cache operators/flags for stores:
\begin{itemize}
\item ``wb'': Default. Cache writeback at all coherent levels. Stores of local data may be cached in L1 or L2, but global data is only cached in L2 since L1s are not kept coherent by hardware. Note that ld.ca's issued by other cores could still hit on stale data.
\item ``cg'': Cache at global level (L2). Data bypasses L1. Same as \verb=st.wb= for global data, for local data marks L1 lines as evict-first.
\item ``cs'': Cache streaming, implying no temporal locality. Allocates in same place as \verb=st.wb= would, but with evict-first policy.
\item ``wt'': Cache write-through to system memory. Applies only to global System Memory addresses to allow a CPU program to poll on the location.
\end{itemize}

Data prefetch is provided by \verb=prefetch= instructions. The level of cache intro which the data should be prefetched is specified explicitly. Prefetch instructions do shared memory addresses do nothing.

\subsubsection{Larrabee}
% "Larrabee: A Many-Core x86 Architecture for Visual Computing". Intel. doi:10.1145/1360612.1360617
Larrabee also adds new instructions and instruction modes for explicit cache control. Examples include instructions to prefetch data into the L1 or L2 caches and instruction modes to reduce the priority of a cache line or evict lines. For example, streaming data typically sweeps existing data out of a cache. Larrabee is able to mark each streaming cache line for early eviction after it is accessed. These cache control instructions also allow the L2 cache to be used similarly to a scratchpad memory, while remaining fully coherent.

\subsubsection{Altivec}
%http://gcc.gnu.org/projects/prefetch.html 
A prefetch instruction specifies one of four data streams, each of which can prefetch up to 128K bytes, 12K bytes in a contiguous block. Reuse of a data stream aborts prefetch of the current data stream and begins a new one. The data stream stop instructions can be used when data from a stream is no longer needed, for example for an early exit of a loop processing array elements.

Additional AltiVec instructions for cache control are lvxl (Load Vector Indexed LRU) and stvxl (Store Vector Indexed LRU), which indicate that an access is likely to be the final one to a cache block and that the address should be treated as least recently used, to allow other data to replace it in the cache.

\subsection{Intel x86 / SSE}

Data prefetch is provided by \verb=prefetch= instructions, which take locality hints (in bits 5:3 of the ModR/M byte) about into which level of the cache the data should be placed. The hints are:
\begin{itemize}
\item ``t0'': Temporal data, fetched into all cache levels
\item ``t1'': Data is temporal with respect to first-level cache, fetched into all levels except ``0th-level'' cache.
\item ``t2'': Data is temporal with respect to second-level cache, fetched into all levels except 0th-level and 1st-level cache.
\item ``nta'': Nontemporal data, fetched into non-temporal cache structure.
\end{itemize}

These hints are processor implementation-dependent, and can be overloaded or ignored by a given processor implementation. The amount of data prefetched is also implementation-dependent, but is at least 32B. The other bits in the ModRM byte are reserved. 

\verb=movntq, movntps=, and \verb=maskmovq= are nontemporal SIMD store variants from register to memory that avoid polluting the cache hierarchy, are no-write-allocate, and are weakly-ordered.

\subsection{VLIW}
\subsubsection{HPL-PD} 
% DOI 10.1.1.13.690
% V. Kathail, M. S. Schlansker, and B. R. Rau. HPL PD architecture specification: Version 1.1. Technical Report HPL-93-80(R.1), Hewlett-Packard, February 2000. http://www.hpl.hp.com/techreports/93/HPL-93-80R1.pdf

Load operations have two modifiers:
{\em Source cache specifiers} used by the compiler to know the estimated data access latency (default is L1). Violation of the latency implied by this modifier means that a stall is required.
{\em Target cache specifiers} used by the processors to indicate the highest level at which data should be kept. Encoded in instruction, but may be ignored.

\subsubsection{Itanium and Itanium 2}
% Reference Manual http://people.freebsd.org/~marcel/refs/ia64/itanium2/25111003.pdf
% Whitepaper  http://www.dig64.org/about/Itanium2_white_paper_public.pdf
% H. Sharangpani and K. Arora. Itanium processor microarchitecture. IEEE Micro, 20(5):24–43, Sept./Oct. 2000. http://ieeexplore.ieee.org/stamp/stamp.jsp?tp=&arnumber=877948
% ISA manual http://www.intel.com/Assets/PDF/manual/324091.pdf

Hint instructions for instruction prefetching: activate/deactivate prefetch engine, special hint on branches.

Ordered loads and stores can be used to force ordering in memory accesses (along with fences).

Bias hint, indicating that SW will modify data in cache line (load as E in MESI coherence).

Explicit data prefetching is done via \verb=lfetch= instruction. Implicit data prefetch is based on the address post-increment of loads, stores, explicit prefetches.

Loads, stores and explicit data prefetches allocate space according to temporal locality hints, which may either case data not to be allocated, or may affect LRU position. The hints are organized according to an abstraction of a N-level memory hierarchy, in which each level contains both a structure for caching data with temporal locality and a structure for caching data with non-temporal locality. An access treated as nontemporal at level N is treated as temporal at level N+1. Obviously the existence of such structures is ``implementation dependent''. Finding a line closer than the hinted distance does not cause demotion.

Example data cache hints:
\begin{itemize}
\item ``NTA'': Nontemporal all levels. Don't allocate in L1, mark as next to replace in L2, don't allocate in L3.
\item ``NT2'': Nontemporal 2 levels. Don't alloc in L1, mark as next in L2, allocate in L3.
\item ``NT1'': Nontemporal 1 levels. Don't alloc in L1, allocate in others.
\item ``T1'': Default, normal allocation in all.
\item ``Bias'': Allocate with intent to modify. L2 and L3 have line in exclusive state.
\end{itemize}

The flush cache instruction \verb=fc= invalidates a particular line in all levels. Write buffers can be flushed with \verb=fwb=.

Cache specifiers are V1 (prefetch cache), and C1-C3(main memory).

The data prefetch cache is used to prefetch large amounts of data having little or no temporal locality without disturbing the conventional first level data cache. In other words, the emphasis in the case of data prefetch cache is more on masking load latencies than on reuse. Accesses to the data prefetch cache don't touch the first-level cache. Prefetch operations are encoded by instructions that load to register 0.

\subsection{RISC}
\subsubsection{PA-RISC}
%http://gcc.gnu.org/projects/prefetch.html 
Some load and store instructions modify the base register, providing either pre-increment or post-increment, and some provide a cache control hint; A load instruction can specify spatial locality, and a store instruction can specify block copy or spatial locality. The spatial locality hint implies that there is poor temporal locality and that the prefetch should not displace existing data in the cache. The block copy hint indicates that the program is likely to store a full cache line of data.

\subsubsection{SPARC}
%http://gcc.gnu.org/projects/prefetch.html 
The SPARC v9 instruction set architecture defines the PREFETCH (Prefetch Data) and PREFETCHA (Prefetch Data from Alternate Space) [15] instructions, whose variants are specified by the fcn field:
\begin{itemize}
\item prefetch for several reads  Move the data into the cache nearest the processor (high degree of temporal locality).
\item prefetch for one read Prefetch with minimal disturbance to the cache (low degree of temporal locality).
\item prefetch for several writes (and possibly reads)  Gain exclusive ownership of the cache line (high degree of temporal locality).
\item prefetch for one write  Prefetch with minimal disturbance to the cache (low degree of temporal locality).
\item prefetch page Shorten the latency of a page fault.
\end{itemize}


\subsubsection{POWER}
%http://www.power.org/resources/downloads/PowerISA_V2.06B_V2_PUBLIC.pdf
PowerPC 603e controls write-back/write-through and caching-enabled on a per page basis. 

Power 2.06 has the following cache control instructions:
\begin{itemize}
\item ``dcbi'': data cache block invalidate
\item ``dcbt'': data cache block touch, data prefetch into a touch buffer, may use Data Stream Control Register to affect HW behavior. (4 straight pages of ``Programming Note'' in the manual)
\item ``dcbtst'': data cache block touch for store, data prefetch, read with intent to modify
\item ``dcbz'': data cache block clear to zero, zeros all bytes, treated as a store
\item ``dbca'': data cache block allocate, allocates undefined space in the cache, treated as a store. Presumably all will be overwritten
\item ``dcbst'': data cache block store, stores the block to memory if it has been modified
\item ``dcbf'': data cache block flush, invalidates unmodified block or writes back modified one
\item ``icbt'': instruction cache block touch, prefetch into instruction cache
\end{itemize}

Power 2.06 has cache locking instructions, which are not hints (cannot be issued speculatively, etc.). The particular cache block being targeted are identified with RA and RB. It is implementation dependent whether coherence invalidate requests and cache control invalidates unlock these cache lines. Overlocking of a given set is also report in an implementation-dependent manner. The instructions are:
\begin{itemize}
\item ``dcbtls'' - Data cache block touch and lock set.
\item ``dcbtstls'' - Data cache block touch for store and lock set.
\item ``icbtls'' - Instruction cache block touch and lock set.
\item ``dcblc/icblc'' - Clear cache block lock in data/instruction cache
\end{itemize}

Some Cache Management instructions contain a 4-bit CT field that is used to specify a cache level within a cache hierarchy or a portion of a cache structure to which the instruction is to be applied. The correspondence between the CT value specified and the cache level is 0 = primary cache, 2 = secondary cache, with other implementation-dependent  options possible.

\subsection{Embedded}
\subsubsection{TI TMS320C6000 / VelociTI}
% ISA focus.ti.com/lit/ug/spru189g/spru189g.pdf
% Cache User's guide http://focus.ti.com/lit/ug/spru656a/spru656a.pdf
% TMS320C621x/671x DSP Two-Level Internal Memory Reference Guide
After a reset, L2cache is disabled and all capacity is L2sram. L2 cache can be enabled in the program code by issuing the appropriate chip support library (CSL) commands. Additionally, in
the linker command file the memory to be used as L2 SRAM has to be specified.
Further, you can control whether external memory addresses are cacheable or noncacheable. Each external memory address space of 16 Mbytes is controlled by one MAR bit (0: noncacheable, 1:cacheable) of the MAR registers
CSL also has support for software-directed invalidates, writebacks or writeback-invalidates from L2 or L1 based on address ranges or for the entire cache. CSL also has routines to set cache mode. All these routines work by writing to special purpose registers. Block/range cache operations execute in the background, allowing other
program accesses to interleave with the block cache operation

\subsubsection{Intel X-SCALE}
%http://download.intel.com/design/intelxscale/31505801.pdf

Can disable and enable the ability of the L1 caches to fill lines. Even when `disabled,' the cache is still checked, but no lines will be filled or evicted. Enabled and disabled via control registers.

Individual lines can be locked in the instruction cache with special instructions that use special registers. For correct operation, the line cannot be already present in the cache, so the line must be explicitly invalidated first, using the same special instruction with a different special register value. Way 0 can never be locked. There is no way to check full-ness, so a table of locked addresses must be maintained by software. The entire cache can be unlocked at once. Invalidating a line also unlocks it.

For the data cache, a mode bit in a special register can be turned on to cause all following loads to be locked in the cached. Locking mode is turned off by another write to the special register.

A RAM can also be created in the data cache using the same locking mode and a separate special register to allocate new lines with unique virtual addresses, instead of fetching existing data.

It is possible to clean, invalidate, or unlock individual cache lines in the data cache, or all lines in the cache at once. Lines can also be allocated in advance of stores, which avoid unnecessary data fetches (deprecated).

Lines can also be locked, unlocked, allocated, cleaned and invalidate in the L2 cache using writes to different special registers. Set- or way-based invalidates of many L2 lines simultaneously are also possible.

As with the L1 cache, software-managed RAM can be allocated and  and deallocated in the L2 cache on a line-by-line basis.

\subsection{Intel SCC}

Flag cache lines with an ``MBPT'' bit in page table, then apply that marking to all cache lines from that page, resulting in them being marked dirty. Dirty cache lines interact correctly with the gets and puts to the message passing buffer.

Per-core LookUp Table for mapping which addresses map to which memory space (i.e. shared DRAM, private DRAM).

\subsection{Academic Proposals}
% Data prefetching http://citeseerx.ist.psu.edu/viewdoc/summary?doi=10.1.1.137.5086
% Multi-module caches. J. Sanchez and A. Gonzalez. A locality sensitive multi-module cache with explicit management. In Proceedings of the 1999 Conference on Supercomputing.  T. Johnson and W.W. Hwu,  “Run-Time  Adaptive Cache Hierarchy Management via Reference Analysis”,  in Procs. of 24th Int. Symp. on Computer Architecture  (ISCA-24), pp. 3 15-326, June 1997

%TODO: Refrences from Wolf and Kandeshar "Memory System Optimization of Embedded Software"

\begin{description}
\item [Rigel]: Rigel LPI supports cache management instructions, explicit software-controlled flushes at the granularity of both the line and the entire cache, memory operations that bypass local caches, and prefetch instructions. \cite{kelm-isca09}
\item [VLS]: Explicit DMA via control registers. Allocation of LS regions via control registers.
%\item[Conditional Kill]: Modifies the LRU position of block or evicts when a condition is met. A cache line kill state is updated only if an access generated by the kill instruction satisfies the cache line offset condition. For example, consider a cache linesize of 4 words (0, ..., 3), an array A, and a reference A[i]. To set the kill state of a cache line when the reference A[i] accesses the fourth word of  the cache line, the load-store instruction corresponding to the reference A[i] would have the cache line offset value 3 as the condition.  So, whenever the reference A[i] accesses the fourth word of  a cache  line,  the kill state of that cache line would be set.  \cite{jain-iccad01}
\end{description}

\section{Proposal Ideas}

Evaluate benefit conferred by specificity?

Interaction with cache coherence protocol: things in SW managed state should move when their thread migrates but remain stationary otherwise. Exclusive and locked, unless owner thread has gone remote? Or nack requests from other threads to enforce protection?

Cache coherence message cost: more expensive to send data than to send messages, unless number of messages is O(P)? Distance may also matter

Dealing with context switches and SW managed data. Is overhead of adding sw managed data to proc state mitigated by fast flush and restore via dma engines?

Coherence is useless without synchronization.

