\chapter{Productive Abstractions of Cache Coherence Policies}
\label{c.coherence}

In a multicore chip with a sizeable hierarchy of on-chip caches,
the majority of the data movement activity that occurs within the chip
is done automatically, in accordance with a cache coherence protocol.
The cache coherence protocol is implemented by the system of cache and memory controllers distributed across the chip.
Since traditional cache coherence protocols preserve the abstraction of a global memory shared by all the cores,
the controllers must work behind-the-scenes to keep copies of data in the right places,
while managing tradeoffs between communication volume, storage capacity and performance.
We predict going forward that, due to the increased percentage of energy consumpution taken up by the memory hierarchy,
that customizable, heterogenous cache coherence policies that minimize data movement for particular use cases
will become an increasingly desirable feature of an on-chip memory hierarchy.
How to define customizable coherence policies, implement the associated protocols efficiently,
and manage the aforementioned communication/performance tradeoff is an important design challenge for future energy-efficient architectures. 

Unfortunately, designing more complex, customizable cache coherence protocols is not a task hardware engineers can undertake lightly.
Protocol correctness can be determined via formal analysis of an abstract model of the protocol and memory system.
However, there are a huge number of ways in which details of the concrete implementation can undermine abstract correctness.
As in any distributed system, modules designed by different teams may interact in unexpected ways,
and assumptions about atomically visible behaviors or event priority levels may be violated,
leading to corrupted data or system deadlock.
Hardware designers shoulder the burden of maintaining the implicit semantics of the abstracted protocol model
throughout the concrete controllers and networks that they build.
This chapter proposes that improving the capabilities of HDLs offers us a path foward to lighten this design burden.
By raising the level of abstraction at which cache controller logic can be described,
and from which synthesizable designs can be generated,
we can smooth over the gap between protocol specification and implementation.

In the previous chapter, I presented TileLink,
a protocol framework designed to be a substrate for cache coherence transactions that implement a particular cache coherence policy within an on-chip memory hierarchy. 
TileLink provides structure in the form of sequences of messages that can be sent between interacting, coherent agents in order
to implement a protocol that is guaranteed to be deadlock free in a nested, hierarchical memory system.
However, TileLink by design says nothing about the particular details of the coherence policy,
which drives the creation and use of these message types.
Filling in details is a task left up to the designers of the cache controllers.

This chapter fills in the aforementioned gaps in the TileLink framework by introducing two further abstractions.
The first is a high-level language,
called \emph{message flows},
taken from the verification literature, 
that describes all the global transactions that make up a particular coherence protocol.
A collection of flows describe every sequence of actions that a protocol can take,
where actions consist of sending TileLink messages and accessing  data and metadata in local memories.
The second abstraction is \emph{coherence metadata objects}.
These objects encapusulate the states that distinguish protocol message flows from one another,
and provide methods for generating TileLink messages and making policy-based decisions within flow transactions.
The abstraction provided by the metadata objects separates the concerns of the controller design from the concerns of the policy design, 
while the TileLink substrate ensures forward progress of global protocol transactions.

\section{Background} 

Designing new cache coherence protocols or supporting a wider variety of more complicated protocols is not a task hardware engineers can undertake lightly.
Verifying the correctness of a cache coherence protocol is a challenging task, and
verifying that a correct protocol has been implemented correctly (using simulations or silicon) is even more difficult
\cite{deorio2008post, bentley2001validating, burckhardt2005verifying, clarke1995verification, dill1992protocol, wood1990verifying}.
Traditionally, protocol correctness is verified using an abstracted version of the distributed system of caches upon which the protocol operates
\cite{talupur2008going, delzanno2003constraint, pong1997verification, wood1990verifying, mcmillan2001parameterized}.
The abstraction employed at this stage makes the verification process tractable by eliding many details of the underlying modules' implementations.
Upon verification of protocol correctness, hardware designers must then use a hardware description language (HDL) to write cache controller logic that correctly implements the protocol.
Unfortuntately, the semantic gap between high-level abstract descriptions of protocols and 
concrete HDL implementations of those same protocols is so wide that verifying the correctness of the protocol
does not come close to guaranteeing the correctness of the final implementation \cite{dave-memocode05}.

I am not the first to propose using a higher level of abstraction to describe cache coherence protocol behavior.
The following approaches each offer a Domain Specific Language (DSL)
built around an abstraction of state machines that perform certain actions when certain conditions are met.
This conditional execution model is a good fit for the requirements of a coherent cache controller,
which must update metadata and data based on a series of messages it sends and receives.
Each high-level description is used to drive the creation of implementations (synthetizable hardware or simulator code),
as well as correctness (verification rules or documentation) from the same source.

Teapot \cite{chandra-dsl97, chandra-sigplan96}
is a Pascal-based DSL for describing coherence protocols using ``continuations'' as an abstraction.
A Teapot program consists of a set of states; each state
specifies a set of message types and the actions to be
taken on receipt of each message, should it arrive for a
cache block in that state.
Teapot provides suspend/resume semantics within each state-based description;
these continuations are used to automatically infer the set of intermediate states and handle unexpected messages.
Continuations in Teapot allow developers to avoid having to manually decompose
a handler into atomically executable pieces and sequence them. 
Teapot outputs C code for distributed memory implementations and $Mur\phi$ models for verification.

Bluespec SystemVerilog (BSV) \cite{bluespec}
is an HDL that produces synthesizeable hardware implementations based on an absraction called guarded atomic actions (GAAs).
BSV has been proposed a a particularly suitable language for describing distributed cache coherence controllers \cite{dave-memocode05}.
GAAs consist of a guard (boolean logic predicate), and an action (some kind of state update)
that is executed atomically by the hardware control logic when the predicate evaluates to true.
Becuase GAAs are also an abstraction that are compatible with many formal verification tools,
and because the BSV compiler produces implementations of rules automatically,
verification overhead for implementations of coherence protocols in this language should be reduced.

The gem5 simulation environment \cite{binkert-sigarch11}
provides SLICC, a DSL for generating state machines for coherence protocols.
SLICC consists of descriptions of individual controller state machines in terms of events, as well as the set of available message types
used to communicate between controllers.
The SLICC compiler outputs C++ simulator code and HTML documentation.

All of the above approaches are based on specifying local descriptions of pieces of a global coherence transaction;
when the state machines they describe are interconnected, the intention is to produce correct global behavior.
This approach reflects a bottom-up philosophy to protocol implementation.
In contrast, we wished to adopt a top-down approach, wherein a global description of transactions
is decomposed into local sub-transactions, which then drive the design of the individual controllers.
We therefore turned to the verification literature to find verification strategies based on expressing global descriptions of protocol behavior.
While many transactional models of coherence protocols have been proposed,
the one best suited to our goals was the \emph{message flow} approach to parameterized procotol verification
\cite{talupur2008going, oleary-fmcad09}.
A message flow is a sequence of messages sent among agents following a protocol that
logically constitutes a single transaction of the protocol. 
In the next section, we discuss how a global, flow-based description of a protocol can be decomposed into a set
of local controller transactions, and discus how we implement those local transactions in Chisel,
our meta-HDL/DSL embedded in Scala.

So far we have discussed prior art in how to describe and implement protocols,
and we now progress from how to build protocols to what protocols to implement. 
Heterogeneity in memory access behavior as been a major focus of study for distributed shared memory systems.
Memory access patterns can general be grouped at a high level into a few common sharing patterns, such as read-modify-write, producer-consumer, and migratory. Systems that support adaptive cache coherence protocols allow the behavior of the protocol to change with detected changes in program behavior.  Examples of such adaptive protocols in the distributed memory space include \cite{amza-hpca97,lebeck-archnews95,stenstrom-isca93,cox-isca93}. Note that these designs have a single protocol, but one that varies its behavior dynamically.

In the shared memory space, designs like FLASH \cite{kuskin-archnews94} have incorporated multiple protocols on top of a single hardware substrate in order to provide adaptability. Generally transitions between protocols have been triggered by heuristic mechanisms \cite{mukherjee-archnews98}. Others \cite{chandra-sigplan96,falsafi-sc94} have proposed creating application-specific protocols that are tailored to match a particular application's needs.
These efforts indicate that multiple protocols can share the same underlying communication framework and memory system,
which served as an inspiration for the TileLink/CoherencePolicy separation of interests described in this chapter and the previous.

\section{Protocol Message Flows}

Talupur and Tuttle showed that message flows are a succinct and
readily available source of high-level invariants that usually go
unused in protocol verification. 
Flows
are often illustrated by protocol designers in the form of
message sequence charts.

Protocol designers use message flows to describe
the basic organization of a protocol and to reason about the
protocol itself.
Message flows implicitly impose constraints
on the order in which the actions appearing within them can happen:
Each action can happen only after the preceding action has been executed

Message flows — similar to message sequence charts — are a natural,
local, linear way for us to think about system behavior in place
of global, monolithic, system-wide invariants

The simplest flows are linear ordering of events usually involving two agents
Each entry in the flow is either a simple event, 
corresponding to a single protocol update being committed or a sub-flow recursively composed of simple events.
The notion of sub-flow allows us to
chop up a complicated flow into smaller units such that each
unit shows interaction between two or more agents engaged in a tightly-coupled causal interaction.
an event might have multiple preceding actions
Similarly, an event in the flow might have more than one succeeding event
Flows may only express a partial order of events and not a total order.
For all of the above reasons, best to represent them as a Directed Acyclic Graph.

Non-interference between flows: which flows are allowed to execute in the system at the same time.
Can be expressed as rules.
In general for our family of coherence protocols, these rules are reflected in the TileLink substrate specified in the previous chapter.
Correct implementations of TileLink by definition enforce the non-interference rules needed by our flows.

\subsection{From Flows to Local Transactions}

Recognizing that a global flow can be divided into sub-flows is useful for providing non-interference lemmas to CMP-based formal verification tools \cite{}.
However, it also provides a mechanism for us to decompose and re-aggregate the contents of sub-flows based on their geographical location.
In other words, each flow touch several distinct agents, and we can collect all those found within an individual agent type in our cache coherence shared memory system.
Then, if we can generate controllers that are capable of performing each sub-flow atomically, as well as send messages to other agents,
we will have furnished ourselves with a substrate that is correct by construction.

In this section we present an algorithm for turning a collection of flows into multiple collections of localized sub-flows.
First, we express the flows as DAGs in Scala, where vertices are events and edges are happens-before dependencies.
Next, we walk each DAG looking for components that are separable sub-graphs of events that all occur at the same agent.
We can sever the graph around these points, leaving us with input vertices that represent receiving a message of a particular type.
These input nodes are the events that kick off local sub-transactions.
Conversely, these mini-DAGs may also contain nodes that require sending a message to anothe agent or agents.

Once we have partitioned all the DAGs, it is trivial to collect the set of sub-flows from each flow that correspond with a particular agent type.
This per-type set then forms the basis of the operations that we will expect this agent to be able to perform atomically.
In the next section we discuss how to turn any of these collections of localized sub-flows into a cache or directory controller.

\subsection{Expressing Localized Sub-Flows and Their Dependencies}

Taking a collection of sub-flows that must be implemented by the particular agent we are designing,
our task is now to 
We do this by creating Transaction Status Handling Registers (TSHRs).
One type of TSHR for each sub-flow, with some capable of merging multiple sub-flows.
In order to make a
provide a way to execute the actions themselves, as well as to implement the dependencies between actions for each flow.

Ideally, we would factor our HDL code such that code describing the common actions and their dependencies are encapsulated,
but made avaulable to each TSHR that needs it.
Scala's traits are perfect for this factoring, as we will show in the following examples.
Each trait consists of functions that actually update state or send a message (i.e., execute a sub-flow).

Update functions related to each port or data structure are provided in specific traits, for example...
Scoreboard of ``pending'' bits track the flow of data and the progress of the transaction.
Dependencies from sub-flow to the next are expressed inside of the trackers themselves,
by referencing the pending bits and connecting each action to the next.
Trackers also contain code to implement the global rules restricting what flows are allowed to execute at the same time.

Examples of tracker code.


\section{Object-Oriented Coherence Policies}

the object-oriented interface
we have defined and made available to implementers of client cache agents, manager cache agents or manager directories. 
By making calls against this object-oriented API, cache controller designers can
create state machines that clearly and correctly implement metadata updates. 
Conversely, designers of new coherence protocols are provided a framework
within which to implement their protocol: filling out the response to each API call.

Metadata objects are the fundamental abstraction used in this API.
These objects are opaque sets of bits which are processed and mutated by the coherence policy API.
Metadata are divided into client-side and manager-side classes,
and any particular cache controller can store either or both types.

Policy as opposed to protocol?

While decomposing flows into traits has proven to be an effective strategy for managaing concurrency and complexity in cache controller design,
this approach says nothing about how different coherence policies are represented within the flows.
For example, what do the state update functions actually store in the metadata arrays?
How are the messages required to be sent between agents actually created?
In this section we prensent an interface address these questions, and moreover to 
hold some parts of the controller design constant, while swapping out only the
elements of controllers that are different between different policiies.
To this end, we created a unified CoherencePolicy interface abstraction that provides the functionality required to implement the specified sub-flows.
Specifically, we built an object-oriented API based around an abstraction of coherence policy metadata.

In the object-oriented programming (OOP) paradigm, \emph{``objects''} are abstractions that contain \emph{fields} of data that are mutated and accessed by procedural \emph{methods}.
In OOP, computer programs are designed by making them out of objects that interact with one another.
In our case, we are forming critical portions of the cache controller logic by interacting with metadata about cache blocks that is stored in local memories.
The specific format of the metadata can be changed without changing the methods that cache controller transactions use,
allowing these aspects to be developed independently, and different metadata implementations to be swapped in easily.


Based on the TileLink structure, we know that their are two kinds of Metadata.
ClientMetadata stores the permissions available to a client in the face of incoming memory operations on a particular block
They may also store protocol-specific information about the block, such as whether or not it has been dirtied by a write.
ManagerMetadata store information about how the block has propagated through the clients for which this manager is responsible.
This might include some representation of the number of client sharers, or patterns of movement observed on that block.

\subsection{Client Metadata} 

A ClientMetadata object consists of a set of bits that abstracts the “state” of a certain cache block,
with respect to the permissions that the policy has made available on that block inside this particular client cache controller.
The metadata may also store other information about the cache block,
for example whether it has been dirtied by a store operation.
The complete API for ClientMetadata can be found in the scaladoc, but we provide a summary here.
There are three types of calls that a controller can make against the API:
permissions checks, message creations, and metadata updates.

\subsubsection{Permissions Checks}

These boolean functions answer questions about the permissions on a cache line, and in particular are used to determine what actions to take relative to specific memory operations.
Memory operation representations are discussed in Appendix~\ref{a.memopcodes}, but the salient feature is that
all of them requires either read or read-and-write permissions.

\begin{description}
\item[isValid()]
Is the block's data present in this cache?
\item[isHit(opcode: UInt)]
Does this cache have permissions on this block sufficient to perform the specified memory operation?
If true, the controller can perform the memory operation immediately.
\item[isMiss(opcode: UInt)]
Does this cache lack permissions on this block sufficient to perform the specified memory op?
If true, the controller needs to initiate a TileLink coherence transaction using makeAcquire().
\item[requiresAcquireOnSecondaryMiss(firstop: UInt, secondop: UInt)]
Does a secondary miss on the block require another Acquire message?
If true, in a controller that supports miss-under-miss transactions, initiate a second coherence transaction using makeAcquire().
\item[requiresReleaseOnCacheControl(opcode: UInt)]
Does a cache control operation (e.g. a voluntary flush) require a Release message to be sent to outer memory?
If true, the control needs to initiate a TileLink coherence transaction using makeVoluntaryRelease.
\item[requiresVoluntaryWriteback()]
Does an eviction caused by a capacity miss require a Release to be sent to outer memory?
If true, the control needs to initiate a TileLink coherence transaction using makeVoluntaryWriteback.
\end{description}


\subsubsection{Message Creation}

These functions return TileLink channel bundles based on the combination of current metadata state and particular memory operations.

\begin{description}
\item[makeAcquire(op\_code: UInt, client\_xact\_id: UInt, addr\_block: UInt) ]
Constructs an Acquire message based on this metdata and a memory operation.
\item[makeVoluntaryRelease(op\_code: UInt, client\_xact\_id: UInt, addr\_block: UInt, addr\_beat: UInt, data: UInt) ]
Constructs a Release message based on this metadata for a cache control op.
\item[makeVoluntaryWriteback(client\_xact\_id: UInt, addr\_block: UInt, addr\_beat: UInt, data: UInt) ]
Constructs a Release message based on this metadata for a capacity eviction.
\item[makeRelease(prb: Probe, addr\_beat: UInt, data: UInt) ]
Constructs a Release message based on this metadata in order to respond to a Probe message from outer memory.
\end{description}

\subsubsection{Metadata Updates}

These functions return new ClientMetadata objects whose internal state has been updated based on a particular coherence event or message received.

\begin{description}
\item[onHit(op\_code: UInt) ]
New metadata after a op\_code hits this block.
\item[onCacheControl(op\_code: UInt) ]
New metadata after op\_code releases permissions on this block.
\item[onProbe(incoming: Probe) ]
New metadata after receiving a Probe message.
\item[onGrant(incoming: Grant, pending: UInt) ]
New metadata after receiving a Grant message in response to the pending memory operation.
\end{description}

\subsubsection{On Reset}

We provide the ClientMetadata.onReset function to initialize any metadata storage arrays within a particular cache.

\subsection{ManagerMetadata API} 

ManagerMetadata is a set of bits that abstracts the “state” of certain cache block,
with respect to the existence of copies of that block in all the client caches
managed by this manager cache controller.
The metadata may also store other information about the cache block,
for example a information about its movement pattern or ownership.
The complete API for MasterMetadata can be found in the scaladoc,
but we provide a summary here.
There are three types of calls that a controller can make against the API:
permissions checks, message creations, and metadata updates.

\subsubsection{Permissions Checks}

These boolean functions answer questions about the permissions on a cache line, and in particular are used to determine what actions to take relative to specific memory operations.

\begin{description}
\item[requiresProbes(acq: Acquire) ]
Does this Acquire require Probes to be sent to the other clients with copies?
\item[requiresProbes(op\_code: UInt) ]
Does this memory operation require Probes to be sent to all clients with copies?
\item[requiresProbesOnVoluntaryWriteback() ]
 Does an eviction caused by a capacity missed require Probes to be sent to all clients with copies?
\end{description}

\subsubsection{Message Creation}

These functions return TileLink channel bundles to use as responses to Clients based on the combination of current metadata state and particular TileLink messages.

\begin{description}
\item[makeProbe(dst: UInt, acq: Acquire) ]
Construct a Probe message based on this metadata in response to a particular Acquire message.
\item[makeProbe(dst: UInt, op\_code: UInt, addr\_block: UInt) ]
Construct a Probe message  based on this metadata in response to a particular cache control operation.
\item[makeProbeForVoluntaryWriteback(dst: UInt, addr\_block: UInt) ]
Construct a Probe message based on this metadata for a capacity eviction.
\item[makeGrant(rel: ReleaseFromSrc, manager\_xact\_id: UInt) ]
Construct an appropriate Grant message to acknowledge a Release message.
\item[makeGrant(acq: AcquireFromSrc, manager\_xact\_id: UInt, addr\_beat: UInt, data: UInt) ]
Construct an appropriate Grant message to acknowledge an Acquire message.
May contain single or multiple beats of data, or just be a permissions upgrade.
\item[makeGrant(pri: AcquireFromSrc, sec: SecondaryMissInfo, manager\_xact\_id: UInt, data: UInt) ]
Construct an appropriate Grant message to acknowledge an Acquire message, overriding some fields
Used to respond to secondary misses merged into this transaction.
May contain single or multiple beats of data.
\end{description}

\subsubsection{Metadata Updates}

These functions return new ManagerMetadata objects whose internal state has been updated based on a particular coherence event or message.

\begin{description}
\item[onRelease(incoming: ReleaseFromSrc) ]
New metadata after receiving a Release message.
\item[onGrant(outgoing: GrantToDst) ]
New metadata after sending a Grant message.
\end{description}

\subsubsection{On Reset}

We provide the ManagerMetadata.onReset function to initialize
any metadata storage arrays within a particular cache.
This function can also be used to generate an empty ManagerMetadata object for API 
use within controllers that do not store any metadata (for example a bus controller).

\subsection{Directory Representation}

We also provide an API for accessing and maintaining directory information on the propagation of cache blocks across all the clients under the purview of a particular manager. These functions are intended to be called from within the CoherencePolicy’s functions, and abstract the details of the storage format used in the directory.

\begin{description}
\item[def pop(prev: UInt, id: UInt): UInt]
Remove id from the prev set of sharers, returning a new set
\item[def push(prev: UInt, id: UInt): UInt]
Add id to the prev set of sharers, returning a new set
\item[def flush: UInt]
Provide an empty set indicating no sharers
\item[def none(s: UInt): Bool]
True if there are no sharers
\item[def one(s: UInt): Bool]
True if there is a single sharer
\item[def count(s: UInt): UInt]
Total count of the sharers
\item[def next(s: UInt): UInt]
Provide the id of the sharer that should be contacted next
\item[def full(s: UInt): UInt]
Provide a full bitmap of all sharers, with 1 indicating a copy.
\end{description}

\subsection{Creating New Coherence Policies}

We give designers planning to implement new coherence protocols several Scala traits
containing abstract members.
These traits are combined to form the \code{CoherencePolicy} interface.
By filling in the missing functions, they can provide a complete coherence policy
that will interoperate with our supplied cache controllers and TileLink networks.
These functions are a superset of the object-oriented APIs discussed in the previous sections,
and are called from within the \code{ClientMetadata} and \code{ManagerMetadata} member methods.

\begin{description}
\item[HasCustomTileLinkMessageTypes] defines the custom, coherence-policy-defined message types, as opposed to the built-in ones. Policies must enumerate the custom messages to be sent over each channel, as well as which of them have associated data.
\item[HasClientSideCoherencePolicy] contains all functions required for client coherence agents. Policies must enumerate the number of client states and define their permissions with respect to memory operations. Policies must fill in functions to control which messages are sent and how metadata is updated in response to coherence events.
\item[HasManagerSideCoherencePolicy] contains all functions required for manager coherence agents. Policies must enumerate the number of manager states. Policies must fill in functions to control which Probe and Grant messages are sent and how  metadata should be updated in response to coherence events.
\end{description}


\section{Hierarchical Translation Between Protocols}

TileLink supports hierarchical nesting of protocols via the Manager-Client Pairing (MCP) framework \cite{beu2011manager}.
In terms of our Coherence 
Hierarchical agent has both ClientMetadata and ManagerMetadata. 
Check the opcode against the ClientMetadata to launch a transaction in the next level.
(TileLink determines the ordering/interleaving of the outer transaction with the inner transaction;
CoherencePolicy's job is to determine which outer transaction is needed.)

We handle translation via a set of pre-defined memory opcodes (see Appendix~\ref{}.).

\section{Concrete Protocols in RocketChip}

We now present a family of five protocols that we have implemented using Chisel, TileLink, CoherencePolicy
and our flow-based cache controllers in the Rocket Chip Generator \cite{rocket}.
These protocols are based around a subset of the classic five state MOESI model
first introduced by Sweazey and Smith \cite{sweazey1986class}.
The protocols contain some subset of the following stable client metadata states~\cite{sorin2011primer}:
\begin{description}
\item[I]. The block is invalid. The cache either does not contain the block or it contains a potentially stale copy that it may not read or write.
\item[M]. The block is valid, exclusive, owned, and potentially dirty. The block may be
read or written. The cache has the only valid copy of the block, the cache must respond to
requests for the block, and the copy of the block at the LLC/memory is potentially stale. 
\item[S].The block is valid but not exclusive, not dirty, and not owned. The cache has a read-only copy of the block. Other caches may have valid, read-only copies of the block.
\item[E].  The block is valid, exclusive, and clean. The cache has a read-only copy of the
block. No other caches have a valid copy of the block, and the copy of the block in the
LLC/memory is up-to-date. 
\end{description}

We have also implemented a more advanced \emph{migratory} protocol. 
This protocol is a reactive protocol, based on proposals by~\cite{stenstrom-isca93,cox-isca93},
that tracks the behavior of cache blocks over time,
identifies migratory behaviors,
and proactively revokes status on the written block in order to make it available to be consumed.
Table~\ref{tab:protocols} lays out the relative capabilities of these five protocols.

\begin{table}[t] 
\begin{center}
\begin{tabular}{|l|c|c|c|c|c|} 
\hline
Name & States & R+W       & RO        & Clean     & Migration \\ \hline
MI        & 2 & \ding{52} &           &           & \\ \hline
MEI       & 3 & \ding{52} &           & \ding{52} & \\ \hline
MSI       & 3 & \ding{52} & \ding{52} &           & \\ \hline
MESI      & 4 & \ding{52} & \ding{52} & \ding{52} & \\ \hline
Migratory & 7 & \ding{52} & \ding{52} & \ding{52} & \ding{52} \\ \hline
\end{tabular}
\caption{Overview of protocols available in Rocket Chip.}
\label{tab:protocols}
\end{center}
\end{table}

Performance measurements, holding design parameters constant (other than core count?).
Figure~\ref{} shows MI vs MEI for read-only data.
Figure~\ref{} shows MI/MEI vs MSI/MESI for shared data.
Figure~\ref{} shows MESI vs Migratory for migratory data.

\section{Discussion}

A coherence {\em protocol} specifies the exact sequences of messages that must be propagated through the memory hierarchy in order to effect a policy.
A coherence {\em policy} governs how the SWMR invariant is represented as metadata indentifying available permissions on data blocks.
TileLink

A protocol consists of many flows, each of which use Policy (for permissions) and TileLink (for ordering).
Common sub-flows can be codified as local transactions, mixed together to form controller logic 
(dependencies among sub-flows can be inferred from flows).

Figure~\ref{} highlights the interplay between message flows, coherence metadata objects, and TileLink.
Policy determines which flow is occuring.
Policy determines set of custom TileLink message types, pick what to send.
Flow determines which policy methods to call.
Flow determines when to send TileLink message.
TileLink determines the possible shapes/ordering/priority of flows (to safe ones).
TileLink determines what types of messages Policy can output.

\section{Directions for Future Work}

Design of custom protocols
We want to allow hooks for software control of coherence states.
Also handled via opcodes targetting individual cache blocks.
Trigger voluntary releases.
Integration with VLS.
See appendix.
Also want to enable software control of which protocol to use, via mode switch or for particular memory regions.

Verification and synthesis from the same description, automatically.
While our flow descriptions should also be compatible with CMP-based tools,
actually automated such a toolchain is not within the scope of this thesis.

Automatic generation of controllers via Bluespec-style rules.
After breaking a set of global message flow transactions into sets of localized rules,
it would be better to automatically infer the control dependencies between them.
Our Chisel description is concise enough that is it easier to reason about than the standard approach,
but still introduces opportunity for human error.
Ideally, we would also automatically infer the contents of CoherencyPolicy functions from a set of decision points extracted from flow descriptions.
For now, they must be filled in manually.

Futher design space exploration.
Multi-level hierarchies.
Energy consumption of data movement.
Storage vs energy tradeoffs: e.g. Directory representations.

\section{Conclusion}

Cache coherence protocol design is one of the toughest challenges in computer architecture.
Through better abstractions, we have attempt to reduce the burden put on hardware devs
to correctly interpret the implicit semantics of abstract coherence models in their implementations.
We propose a top-down approach to protocol specification based on message flows, and provided a strategy
for transforming such specifications into Chisel implementations of cache controllers.
We also utilize object-oriented programming to encapsulate the policy-specific decisions encoded
in the flows, making it easy to swap policies without changing any controller code.
Opens the door to more flexible and customizable protocol design, which will be important
to the future of energy-efficiency in the on-chip memory hierarchy.
